\titre{Ouvrez, ouvrez la cage aux papiers.}

Tu as peut-être déjà remarqué que depuis l'ENS, on a accès à beaucoup d'articles dans des revues. Ces abonnements en ligne, principalement mutualisés avec 264 autres universités, pèsent lourd sur le budget de la recherche. Ces coûts sont hérités de l'époque où il était cher d'imprimer et diffuser des revues, si bien que l'industrie de l'édition académique dégage aujourd'hui de larges bénéfices sur le dos des universités.

Pour changer cela, le mouvement de l'open access propose de mettre en ligne les articles dans des dépôts publics. Beaucoup d'universités encouragent déjà leurs chercheur-e-s à le faire, avec plus ou moins de succès. Une équipe s'est formée pour que l'ENS prenne position dans ce sens, avec succès. L'administration s'est montrée enthousiaste, il faut maintenant trouver un moyen de convaincre les départements de libérer leurs articles, et les y aider.

On a donc commencé à recenser automatiquement les papiers écrits par les chercheurs de l'ENS, et à analyser s'ils sont déjà librement disponibles ou non.
Tu peux aller voir notre prototype (très imparfait) sur \url{dissem.in}. 

L'idée c'est que, si ce site et le projet te plaisent, tu nous rejoignes ! Si tu n'es pas informaticien-ne, ça tombe bien ! On a vraiment besoin de toi, pour suggérer des améliorations, pour motiver tes collègues de département, et nous aider à organiser des évènements pour l'Open Access Week en Octobre ! Tous les niveaux d'implication sont bienvenus. Tu peux t'inscrire à notre mailing list à annonces-inscription@dissem.in.

\signature{pintoch et bThom pour l'équipe de dissemin - team@dissem.in}
