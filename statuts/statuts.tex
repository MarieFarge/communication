\documentclass[a4paper]{article}

\usepackage[T1]{fontenc}
\usepackage[utf8]{inputenc}
\usepackage{fullpage}

\usepackage[french]{babel}

\title{Statuts de l'association C.A.P.S.H.}
\date{5 septembre 2015}

\begin{document}

\maketitle

\subsection*{Article premier. - Nom}

Il est fondé entre les adhérents aux présents statuts une association régie par la loi du 1er juillet 1901 et le décret du 16 août 1901, ayant pour titre Comité pour l'Accessibilité aux Publications en Sciences et Humanités (soit C.A.P.S.H), ou en anglais
Committee for the Accessibility of Publications in Sciences and Humanities.
Par métonymie avec un de ses projets, l'association peut également être désignée par "Dissemin" ou "dissem.in".

\subsection*{Article 2. - Objet}

Cette association a pour objet la promotion de l'accès libre aux publications académiques et
aux données de recherche, le développement et la mise en place d'une infrastructure logicielle pour faciliter l'accès à ces données, et le financement de ces activités par la collecte de dons et de fonds publics ou privés.
Cette association peut vendre les services suivants à des utilisateurs :
\begin{enumerate}
    \item l'analyse de l'état d'accessibilité d'un ensemble de publications ou données ;
    \item le conseil sur les mesures à prendre pour améliorer cette accessibilité ;
    \item le dépôt mandaté de publications ou données pour le compte de leurs auteurs ;
    \item le développement de fonctionnalités logicielles destinées aux besoins spécifiques de certains utilisateurs, au sein des logiciels développés par l'association ;
    \item la maintenance d'une infrastructure informatique destinée spécifiquement à certains utilisateurs, pour les logiciels développés par l'association ou les prestations ci-dessus ;
    \item l'accès à une version privilégiée d'un site Web ou autre ressource en ligne éditée par l'association qui offre des fonctionnalités supplémentaires.
\end{enumerate}

\subsection*{Article 3. - Siège social}

Le siège social est fixé au ...
Il pourra être transféré par simple décision de l'assemblée générale.

\subsection*{Article 4. - Durée}

La durée de l'association est illimitée.

\subsection*{Article 5. - Composition}

L'association se compose de membres, qui sont des personnes physiques ou morales partageant les objectifs de l'association, répartis dans deux catégories:
\begin{itemize}
  \item les membres actifs, qui s'impliquent dans les activités de l'association ;
  \item les membres associés, qui se proposent de conseiller l'association dans ses orientations, mais n'ont pas de pouvoir décisionnel.
\end{itemize}

\subsection*{Article 6. - Admission}

Toute personne physique ou morale peut devenir membre actif ou membre associé, sous réserve que son adhésion soit approuvée
par le conseil d'administration. Celui-ci dispose d'un délai de deux mois pour statuer sur la demande.

\subsection*{Article 7. - Membres, cotisations}

Aucune cotisation n'est exigée, les membres étant libres d'effectuer des dons ponctuels à leur convenance.

\subsection*{Article 8. - Radiations}

La qualité de membre se perd par:
\begin{enumerate}
    \item La démission ;
    \item Le décès ;
    \item La radiation prononcée par le conseil d'administration pour motif grave, l'intéressé·e ayant été
        invité·e à fournir des explications par écrit.
\end{enumerate}

\subsection*{Article 9. - Affiliation}

La présente association peut adhérer à d'autres associations, unions ou regroupements par décision du
conseil d'administration.

\subsection*{Article 10. - Ressources}

Les ressources de l'association comprennent:
\begin{itemize}
    \item Les dons ponctuels, provenant de ses membres ou de toute autre personne ;
    \item Les subventions d'établissement publics ou privés ;
    \item Le produit de ses services facturés à ses utilisateurs ;
    \item Toute autre ressource autorisée par 
        les lois et règlements en vigueur.
\end{itemize}

\subsection*{Article 11. - Conseil d'administration}

\paragraph{Constitution}
Le conseil d'administration (CA) est composé d'au moins deux
personnes physiques membres de l'association.
Il est investi de tous les pouvoirs nécessaires au fonctionnement de l'association, et prend toutes les décisions portant sur l'association.
Les décisions sont prises par délibérations entre les membres du CA, les désaccords éventuels étant tranchés par un vote à la majorité.
Les membres du CA assument conjointement et solidairement les rôles et responsabilités de président, trésorier et secrétaire de l'association.

\paragraph{Mandats}
Le CA est libre de donner ou de révoquer mandat à des 
membres de l'association
pour accomplir des tâches nécessaires à la réalisation de son objet.

\paragraph{Renouvellement}
Le renouvellement des membres du CA est effectué annuellement par l'assemblée générale ordinaire suivant les modalités prévues à l'article 12.


\subsection*{Article 12. - Assemblée générale ordinaire}

\paragraph{Objet}
Chaque année, dans le mois suivant la date anniversaire de la création de l'association,
le CA convoque une assemblée générale ordinaire. Celle-ci permet au CA d'exposer aux membres la situation
morale et financière de l'association, permet aux membres de délibérer sur tout sujet inscrit à l'ordre du jour, et permet de procéder au renouvellement des membres du CA.

\paragraph{Constitution} L'assemblée générale ordinaire est ouverte à
 tous les membres de l'association. Seuls les membres actifs y ont voix délibérative, les membres associés n'y jouent qu'un rôle consultatif.

\paragraph{Convocation}
L'assemblée générale ordinaire est convoquée par le CA, de sa propre initiative ou sur demande d'un membre actif.
Au moins une semaine avant la date fixée, les membres de l'association sont convoqués par le CA. L'ordre du jour figure sur les convocations.

\paragraph{Déroulement}
Le CA préside l'assemblée. 
L'assemblée générale ne peut délibérer que sur les points inscrits à l'ordre du jour.
Il est procédé, après épuisement de l'ordre du jour, au renouvellement des membres du conseil
d'administration, qui peuvent être reconduits dans leurs fonctions. Pour chaque membre du CA, l'assemblée générale délibère sur la reconduction ou la révocation de son appartenance au CA. Elle délibère ensuite sur l'introduction au CA de tout membre proposé par l'assemblée générale.

\paragraph{Délibération}
Les décisions sont prises à la majorité des voix des membres actifs présent·e·s ou représenté·e·s. Chaque
membre peut représenter au plus deux autres membres.
Pour que le vote soit valide, il est nécessaire qu'au moins le quart des membres actifs se soient exprimé·e·s.
Les décisions de l'assemblée générale s'imposent à tous les membres, y compris absents ou
représenté·e·s.

\subsection*{Article 13. Assemblée générale extraordinaire}

Une assemblée générale extraordinaire peut être convoquée à la demande de la majorité
des membres actifs.
Les modalités de convocation sont les mêmes que pour l'assemblée générale ordinaire.
Elle ne peut se prononcer que sur la
dissolution de l'association ou la modification de ses statuts.

\subsection*{Article 14. - Indemnités}

Toutes les fonctions, y compris celles des membres du CA, sont gratuites et
bénévoles. Seuls les frais occasionnés par l'accomplissement de leur mandat sont
remboursés sur justificatifs. Le rapport financier présenté à l'assemblée générale
ordinaire fait état, par bénéficiaire, des remboursements de frais de mission,
de déplacement ou de représentation.

\subsection*{Article 15. - Règlement intérieur}

Un règlement intérieur peut être établi par le CA, qui le fait alors
approuver par l'assemblée générale.
Ce règlement est destiné à fixer les divers points non prévus par les présents statuts,
notamment ceux qui ont trait à l'administration interne de l'association.

\subsection*{Article 16. - Dissolution}

En cas de dissolution prononcée selon les modalités prévues à l'article 13, un ou plusieurs
liquidateurs sont nommés, et l'actif, s'il y a lieu, est dévolu conformément aux décisions
prises lors de l'assemblée générale extraordinaire qui statue sur la dissolution.

\end{document}

