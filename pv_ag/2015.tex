\documentclass[a4paper]{article}

\usepackage[T1]{fontenc}
\usepackage[utf8]{inputenc}
\usepackage{fullpage}

\usepackage[official]{eurosym}
\usepackage[french]{babel}

\title{Procès-verbal de l'assemblée générale constitutive}
\date{5 septembre 2015}

\begin{document}

\maketitle

Au cours de la soidée du samedi 5 septembre 2015, les fondateurs de l'association C.A.P.S.H. se sont réunis en assemblée générale sur Internet, par le biais du canal de discussion \#openaccess du serveur IRC ulminfo.fr. La séance est ouverte à 22h15.

Sont présents (adresses omises dans cette version publique):
\begin{itemize}
  \item Antoine Amarilli
  \item Thomas Bourgeat
  \item Antonin Delpeuch
  \item Marie Farge
  \item Pablo Rauzy
\end{itemize}
    
L'assemblée générale désigne Antonin Delpeuch en qualité de président de séance et Antoine Amarilli en qualité de secrétaire de séance.
Le président de séance met à la disposition des présents le projet de statuts de l'association et l'état des actes passés pour le compte de l'association en formation.

Puis il rappelle que l'assemblée générale constitutive est appelée à statuer sur l'ordre du jour suivant :
\begin{itemize}
  \item présentation et correction du projet de statuts de l'association ;
  \item adoption des statuts ;
  \item désignation des premiers membres du conseil d'administration ;
  \item reprise des actes passés pour le compte de l'association en formation ;
  \item choix d'un compte courant et désignation des personnes autorisées à faire fonctionner le compte
  \item pouvoirs en vue des formalités de déclaration et publication.
\end{itemize}
    
\paragraph{1ère délibération} Marie Farge propose un changement au nom de l'association. Plusieurs propositions sont mentionnées. La proposition finale est d'adopter le nom "Comité pour l'Accessibilité aux Publications en Sciences et Humanités", avec les autres noms C.A.P.S.H. et le nom anglais "Committee for the Accessibility of Publications in Sciences and Humanities". La proposition est mise au vote et est adoptée à l'unanimité.

\paragraph{2ème délibération} Antonin Delpeuch propose de choisir "Paris (75)" comme siège social pour l'association. Il est décidé qu'il est nécessaire de choisir une localisation plus précise. Antonin Delpeuch propose alors son adresse personnelle, "34 rue de la Chanaise, 71250 Cluny". Une discussion s'engage sur l'influence de ce choix sur l'ouverture d'un compte, mais il est décidé que la question sera étudiée plus tard. La proposition est donc mise au vote, et adoptée à l'unanimité.

\paragraph{3ème délibération} La question est posée du choix de la banque. Il est proposé d'ouvrir un compte courant au Crédit Agricole Paris, qui propose une offre sans frais mensuels de tenue de compte pour les associations. Il est débattu de si la domiciliation de l'association à Cluny est compatible avec l'ouverture d'un compte auprès de la branche parisienne du Crédit Agricole. Il est décidé de maintenir la domiciliation actuelle, et, si le Crédit Agricole refuse de nous ouvrir un compte, d'ouvrir un compte à la Banque Postale. La proposition est mise au vote et est adoptée à l'unanimité. Dans les deux cas, les membres du CA suivants seront habilités à faire fonctionner le compte :
\begin{itemize}
  \item Antoine Amarilli
  \item Antonin Delpeuch
\end{itemize}
Cette liste est mise au vote et adoptée à l'unanimité.

\paragraph{4ème délibération} L'assemblée générale constitutive désigne en qualité de premiers membres du conseil d'administration (adresses et emplois omis dans cette version) :
\begin{itemize}
  \item Antoine Amarilli
  \item Thomas Bourgeat
  \item Antonin Delpeuch
  \item Marie Farge
  \item Pablo Rauzy

\end{itemize}
La liste est mise au vote et est adoptée à l'unanimité.

Conformément aux statuts, cette désignation est faite pour une durée expirant lors de l'assemblée générale qui sera appelée à statuer sur les comptes de l'exercice clos le 31 août 2016. Les membres du CA ainsi désignés acceptent leurs fonctions.

\paragraph{5ème délibération} Antonin rend compte des services auxquels il a souscrit à titre personnel et dont le transfert à l'association est envisagé :
\begin{itemize}
  \item Deux noms de domaine souscris auprès de GANDI SAS:
    \begin{itemize}
        \item "dissem.in", dont la réservation expire le 3 décembre 2015 et dont le renouvellement coûte 7,50\euro{} TTC par an
        \item "dissemin.net", dont la réservation expire le 29 mars 2016 et dont le renouvellement coûte 16,80\euro{} TTC par an
    \end{itemize}
\item Le serveur dédié Dedibox XC loué auprès de Online SAS, pour un montant de 19,19\euro{} TTC par mois, dont la prochaine échéance sera facturée pour le mois de septembre 2015 à la fin de ce mois.
\end{itemize}
    
Antonin fait part de son intention d'effectuer les dons ponctuels nécessaires pour couvrir ces frais en l'absence d'autres ressources. Chacun des quatre autres membres présents fait part de sa volonté d'aider financièrement en cas de besoin. Le transfert des services à l'association est mis au vote et est adopté à l'unanimité.

\paragraph{6ème délibération} Les statuts sont relus par chacun, et les corrections proposées sont effectuées. L'état final des statuts est mis au vote. Les statuts sont adoptés à l'unanimité.

\paragraph{7ème délibération} Antonin se propose pour se charger des formalités administratives de déclaration de l'association. Cette proposition est mise au vote et acceptée par l'ensemble des présents.

\medskip
L'ordre du jour étant épuisé, la séance est levée à 23h10.

\end{document}

