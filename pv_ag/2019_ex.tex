\documentclass{scrartcl}
\usepackage[francais]{babel}
\usepackage[utf8]{inputenc}
\usepackage{hyperref}

\title{Compte-rendu de l'assemblée générale extraordinaire de l'association C.A.P.S.H.}
\author{Antoine Amarilli}
\date{24 mai 2019}

\hyphenation{ORCID}

\begin{document}
\maketitle

Une assemblée générale extraordinaire de l'association C.A.P.S.H.\ s'est tenue
le 24 mai 2019 sur Internet, par le biais du canal de discussion \#openaccess du
serveur IRC ulminfo.fr. Conformément à l'article 13 des statuts de
l'association, celle-ci a été convoquée à la demande de la majorité des membres
actifs, à savoir : 

\begin{itemize}
  \item Antoine Amarilli (membre du CA)
  \item Antonin Delpeuch (membre du CA)
  \item Pablo Rauzy (membre du CA)
  \item Lucas Verney
\end{itemize}
% 
Sont présents :

\begin{itemize}
\item Antoine Amarilli
\item Pablo Rauzy
\item Lucas Verney
\end{itemize}
%
L'unique point figurant à l'ordre du jour, et communiqué aux membres de
l'association lors de la convocation de l'AG, est le suivant :

\begin{itemize}
\item Changement des statuts : clarification de la dévolution des biens en cas
  de dissolution (article 16)
\end{itemize}
%
La séance est ouverte à 19h10. Antoine Amarilli se propose pour endosser la
responsabilité de président de séance : cette proposition est mise au vote et
recueille l'unanimité des voix.

\paragraph{Changement des statuts.} 
Antoine Amarilli rappelle que l'association a reçu une réponse le 25 avril à son
rescrit fiscal visant à déterminer si celle-ci est éligible au régime des dons
mentionnés aux articles 200 et 238-bis du code général des impôts. Cette réponse
fait état de la réserve suivante :
\begin{quote}
  \og Cependant, il convient de préciser que
vos statuts ne font pas état de la dévolution des biens en cas de dissolution
volontaire ou statutaire de votre association, or l'article 11 du décret du 16
août 1901 modifié par l'article 1 du décret 2017-908 du 6 mai 2017 vous fait
obligation d'indiquer dans vos statuts l'attribution des actifs de votre
association. \fg
\end{quote}

Pour cette raison, Antoine Amarilli propose le changement suivant des statuts :

\begin{quote}
\og Le contenu de l'article 16 ``Dissolution'' est remplacé par le texte suivant
: ``En cas de dissolution prononcée selon les modalités prévues à l’article
12, un ou plusieurs liquidateurs sont nommés, et l'actif net, s'il y a lieu, est
dévolu à un organisme ayant un but non lucratif ou à une association ayant des
buts similaires conformément aux décisions de l’assemblée générale
extraordinaire qui statue sur la dissolution. L’actif net ne peut être dévolu à
un membre de l’association, même partiellement, sauf reprise d’un apport. Les
dispositions de cet article s'appliquent à la dissolution volontaire,
statutaire, prononcée en justice, par décret, ainsi qu'à tout autre motif de
dissolution.'' \fg
\end{quote}

Les membres présents, informés à l'avance du projet de modifications,
l'approuvent sans réserve. Mise au vote, la proposition recueille l'unanimité
des suffrages. Elle est donc adoptée, et les statuts de l'association sont
modifiés en conséquence. Antoine indique qu'il se chargera de la mise à jour des
statuts sur le site de l'association, de la déclaration en préfecture, et de la
réponse au rescrit fiscal.

\bigskip

L'ordre du jour étant épuisé, la séance est levée à 19h20.
\end{document}

