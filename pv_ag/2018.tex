\documentclass{scrartcl}
\usepackage[francais]{babel}
\usepackage[utf8]{inputenc}

\title{Compte-rendu de l'assemblée générale ordinaire de l'association C.A.P.S.H.}
\author{Antoine Amarilli}
\date{2 septembre 2018}

\hyphenation{ORCID}

\begin{document}
\maketitle

L'assemblée générale de l'association C.A.P.S.H.\ s'est tenue au cours de la
soirée du dimanche 2 septembre 2018 au domicile de Marie Farge à Paris, à partir
de 18:25.
% 
Sont présents :

\medskip

\begin{itemize}
\item Antoine Amarilli
\item Antonin Delpeuch
\item Marie Farge
\item Ryan Lahfa
\item Patricia Mirabile
\item Pablo Rauzy
\end{itemize}

\medskip

Patrica Mirabile endosse la responsabilité de présidente de séance et Antoine
Amarilli celle de secrétaire de séance, ce que l'assemblée approuve à l'unanimité.
Les points suivants figurent à l'ordre
du jour :

\medskip

\begin{itemize}
\item Bilan moral
\item Bilan financier
\item Rescrit fiscal pour le caractère d'intérêt général de l'association
\item Financement de voyages
\item ANR
\item COSO
\item État des négociations Couperin
\item Réponse au suivi de l'appel à projets PSL Explore
\end{itemize}

\paragraph{Bilan moral.} 
Le bilan moral est présenté par Antonin Delpeuch.
Il reprend les statistiques calculées, qui établit qu'il y a eu de l'ordre de
16\,000 actions de dépôt en 2017-2018.
Il mentionne son travail avec Federico Leva pour le développement d'OAbot et
rendre les citations sur Wikipédia accessibles. Il mentionne l'intérêt exprimé
par un utilisateur pour l'API de Dissemin.

En termes de développment, il y a eu peu d'activité, mais la plateforme est
maintenue. Il y a eu quelques contributions, notamment en termes de
localisation dans un grand nombre de langues grâce à l'effort de contributeurs
bénévoles. Il serait possible sans difficulté d'activer davantage de langues. Il
est maintenant possible d'ajouter ou de retirer des papiers de son profil
Dissemin, sans que ce soit répercuté sur ORCID. Il manque toujours la
fonctionnalité d'édition.

Marie Farge explique qu'elle a continué à faire la promotion de Dissemin à
différentes conférences, et à en faire des démonstrations. Il est proposé de
centraliser la liste de ces présentations sur le site Web et dans le rapport
d'activité.

Patricia rappelle qu'elle a envoyé une newsletter interne pour centraliser les
activités de l'association.

Pablo explique qu'il a participé au dictionnaire des communs.

Le bilan moral est approuvé à l'unanimité.

\paragraph{Bilan financier.} Le bilan financier est présenté par Antoine
Amarilli. Il s'appuie sur des documents préalablement portés à la connaissance des
membres de l'AG par voie électronique, et dont la teneur suit.

Au 2 septembre, le solde du compte bancaire de l'association se porte à
4370.54~EUR. Ses recettes mensuelles (dons récurrents) se portent à 10~EUR,
ainsi que 2.96~EUR par semaine environ sur la plateforme Liberapay (non
comptabilisées dans le bilan). Les dépenses mensuelles de l'association se
portent à 86.36~EUR, et peuvent donc être couvertes pour encore 50 mois avec nos
fonds actuels même sans autre rentrée d'argent.
 
Sur la période courant du 1$^{\mathrm{er}}$ septembre 2017 au 31 août 2018, les
recettes de l'association se portent à 253.23~EUR, dont :

\begin{itemize}
  \item 120~EUR de dons mensuels récurrents (10~EUR sur 12 mois)
  \item 40~EUR de dons mensuels (5~EUR sur 8 mois)
  \item 93.23~EUR de dons collectés via Liberapay
\end{itemize}

Les dépenses sont de 1101.72~EUR, dont :

\begin{itemize}
  \item 1036.32~EUR de location de serveurs auprès de Online (86.36~EUR sur 12
    mois)
  \item 65.40~EUR par chèque le 15 janvier
\end{itemize}

En plus de ses recettes sur le compte en banque, l'association dispose toujours
de la majeure partie du financement PSL Explore d'un montant de 15440~EUR, qui a
été utilisé pour financer des missions. 
On estime qu'il reste 12185~EUR sur ce contrat.
L'argent restant est disponible jusqu'au
31 décembre 2021. Malheureusement, l'utilisation de cet argent nécessite
beaucoup de démarches...

Les moyens de paiement de l'association (chéquier) sont conservés par Antoine.

Une discussion s'engage au sujet de l'utilisation de l'argent PSL.

Le bilan financier est approuvé à l'unanimité.

\paragraph{Rescrit fiscal pour le caractère d'intérêt général de l'association.}
Est évoquée la question de faire en sorte que les dons effectués à Dissemin
soient déduisibles des impôts. Après étude, il semblerait que CAPSH puisse avoir
le statut d'association d'intérêt général. Il faut effectuer une procédure de
rescrit fiscal pour déterminer cela.

Antoine Amarilli se propose pour étudier cette question.

\paragraph{Financement de voyages.}
Marie Farge présente des missions pour laquelle elle souhaiterait être
remboursée par PSL Explore pour présenter Dissemin. L'association approuve cette
utilisation des fonds de PSL Explore.

\paragraph{ANR.}
Nous remarquons que l'ANR, dans son AAPG 2019, mentionne (Section C.4) que les
articles issus de projets financés doivent être déposés dans des archives.
L'assemblée générale discute de la pertinence, d'une part de s'adresser à l'ANR
pour leur présenter Dissemin et les encourager à en mentionner l'existence
auprès des projets financés, d'autre part de s'adresser aux membres de projets
ANR pour leur présenter Dissemin comme moyen de satisfaire à leurs obligations
vis-à-vis de l'ANR.

\paragraph{COSO.}
Plusieurs membres de l'association ont candidaté au comité pour la science
ouverte (COSO) : Patricia, Antonin, et Pablo ont candidaté. Patricia est en
cours de discussion avec les organisateurs.

\paragraph{État des négociations Couperin.}
L'assemblée générale discute de l'état actuel des négociations entre COUPERIN et
les publishers.

\paragraph{Réponse au suivi de l'appel à projets PSL Explore.}
L'assemblée générale prépare la rédaction de la réponse au suivi de l'appel à
projets PSL Explore.

\bigskip

L'ordre du jour étant épuisé, la séance est levée à 21h.
\end{document}

