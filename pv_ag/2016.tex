\documentclass{scrartcl}
\usepackage[francais]{babel}
\usepackage[utf8]{inputenc}

\title{Compte-rendu de l'assemblée générale ordinaire de l'association C.A.P.S.H.}
\author{Antoine Amarilli}
\date{20 septembre 2016}

\hyphenation{ORCID}

\begin{document}
\maketitle

L'assemblée générale de l'association C.A.P.S.H.\ s'est tenue au cours de la
soirée du lundi 20 septembre 2016 au domicile d'Antoine Amarilli à Montrouge
(92), à partir de 19h.
% 
Sont présents :

\medskip

\begin{itemize}
\item Antoine Amarilli
\item Marie Farge
\item Antonin Delpeuch
\item Marc Jeanmougin
\item Pablo Rauzy
\item Raito Bezarius
\item Patricia Mirabile
\end{itemize}

\medskip

Antonin Delpeuch endosse la responsabilité de président de séance et Antoine
Amarilli celle de secrétaire de séance. Les points suivants figurent à l'ordre
du jour :

\medskip

\begin{itemize}
\item Bilan moral
\item Bilan financier
\item Partenariat CCSD/Couperin
\item Résultats du financement PSL
\item Relations avec Academia.edu
\item Relations avec ORCID
\item Gestion des articles scientifiques manquants
\item Organisation centralisée ou fédérée
\item OAbot
\item OA Week
\item Démonstration du dépôt dans HAL
\item Open Science Conference
\item Listes de diffusion
\item Point sur la loi Lemaire
\item Dissemin et données ouvertes
\item Plateformes de discussion
\item Organisation de réunions régulières
\end{itemize}

\paragraph{Bilan moral.} Le bilan moral est présenté par Antonin Delpeuch. Il
fait brièvement état des principales actions entreprises par l'association au
cours de l'année écoulée, et de leurs retombées. Le bilan est tacitement
approuvé par les membres présents.

\paragraph{Bilan financier.} Le bilan financier est présenté par Antoine
Amarilli. Il s'appuie sur des documents préalablement portés à la connaissance des
membres de l'AG par voie électronique, et dont la teneur suit.

Au 12 septembre, le solde du compte bancaire de l'association se porte à 183.99 EUR. 
Ses recettes mensuelles (dons récurrents) se portent à 15 EUR, et ses dépenses
mensuelles à 86.36 EUR.

Sur la période courant du 1$^{\mathrm{er}}$ septembre 2015 au 31 août 2016, les recettes
totales se montent à 877.18~EUR, dont 155~EUR de dons récurrents (deux
donateurs), et 722.18~EUR de dons ponctuels reçus par virement bancaire : le
\mbox{2015-10-05} (50~EUR),
\mbox{2015-10-22} (300~EUR),
\mbox{2015-10-29} (67.75~EUR),
\mbox{2015-11-02} (27.18~EUR),
\mbox{2015-11-10} (30~EUR),
\mbox{2015-11-25} (97.25~EUR),
\mbox{2016-02-18} (150~EUR).

Les dépenses totales se montent à 611.83 EUR, dont une dépense ponctuelle de
32.40~EUR pour renouvellement de nom de domaine auprès de Gandi, et des dépenses
totales de 579.43~EUR de location de serveurs auprès de Online.

\paragraph{Partenariat CCSD/Couperin.} Interrogé sur la santé financière de
l'association, Antonin Delpeuch rappelle que le partenariat avec le CCSD, déjà
signé, prévoit un financement de 9\,000 EUR pour l'association, dont le
versement devrait intervenir prochainement. Il est calculé que ce financement
pourra couvrir les frais de fonctionnement actuels de l'association, sur la base
des dons récurrents actuels, pour une période de plus de dix ans.

Pour réduire les dépenses, Antonin Delpeuch mentionne également la possibilité de renoncer
à la location d'un des deux petits serveurs actuellement loués par
l'association, une fois que la migration des services vers le nouveau serveur
(de plus forte capacité) aura été accomplie.

\paragraph{Résultats du financement PSL.} L'AG regrette de ne pas avoir encore
eu connaissance des résultats de la candidature de l'association à l'appel à
projet PSL Explore, et reporte la discussion de ce point à une date ultérieure.

\paragraph{Relations avec Academia.edu.} L'AG salue l'ouverture par Academia.edu
d'une interface OAI-PMH permettant le moissonage d'articles. Antonin Delpeuch
rappelle que Academia.edu a également choisi de permettre aux utilisateurs
atteignant une de leurs pages depuis Wikipédia d'accéder au texte intégral des
articles hébergés sans avoir besoin de se créer un compte.

Antonin Delpeuch mentionne la situation avec ResearchGate, dont le crawl n'est
pas fonctionnel à l'heure actuelle.

\paragraph{Relations avec ORCID.}
Marie Farge mentionne les discussions qu'elle a menées avec ORCID. Certaines réserves
sont émises au sujet de la collecte d'informations, qu'ils entreprennent suivant
différents fronts (par exemple, source de financements), et sur leur statut de
\emph{not-for-profit} qui ne semble toutefois pas les mettre entièrement à
l'abri d'une appropriation commerciale. Sont toutefois mentionnées les garanties
de gouvernance rapportées à Marie Farge. L'importance d'ORCID et l'existence
de solutions alternatives sont discutées, mais Marie Farge rappelle qu'ORCID est
la solution commune vers laquelle ont convergé les acteurs existants, et qu'il
apparaît donc difficile de la remplacer.

En ce qui concerne l'utilisation d'ORCID par Dissemin, Antonin Delpeuch rappelle
qu'ORCID est utilisé pour externaliser le problème de l'authentification des
chercheurs, et celui de la désambiguation d'articles : suivant le plan actuel,
les utilisateurs actuels qui ne sont pas associés à un compte ORCID pourront
simplement énumérer leurs articles par le biais d'une recherche textuelle, avec
de possibles problèmes d'homonymie. Une fois associés à un compte ORCID, le plan
prévoit de permettre aux utilisateurs de modifier leur liste de publications
sans les renvoyer vers l'interface ORCID tierce pour ce faire. En revanche, ceci
ne semble pas possible sans devenir membre d'ORCID. Après étude des tarifs
proposés sur le site d'ORCID, l'AG estime qu'une adhésion à ORCID n'est pas
financièrement viable aujourd'hui. Antonin Delpeuch fait état de l'échec de
tentatives antérieures d'obtenir malgré tout un accès préférentiel aux fonctionnalités
nécessaires.

L'AG discute la possibilité d'une adhésion à ORCID par Couperin, qui pourrait
alors faire bénéficier Dissemin de leur accès. Malheureusement, l'AG estime que
la structure complexe de Couperin, et les montants et délais en jeu, font que
cette perspective ne pourra probablement pas régler le problème à court terme.

Antoine Amarilli fait remarquer le déséquilibre apparent de la situation
actuelle, où l'association devrait payer ORCID pour avoir le droit de leur
fournir des données. L'AG demande si d'autres acteurs se trouvent dans une
situation analogue à la nôtre. Antonin Delpeuch rappelle que les outils de
gestion de recherche vendus aux universités les placent dans une sitation
similaire à la nôtre ; à leur échelle, cependant, le montant de la cotisation à
ORCID ne représente pas un problème de la même ampleur.

L'AG propose ainsi une solution où les éditions de publications par les
utilisateurs seraient conservées localement par Dissemin, mais ne seraient pas
propagées à ORCID, faute d'une interface pour ce faire. L'AG mentionne également
la possibilité de permettre aux utilisateurs une intégration manuelle de leurs
données dans ORCID, par exemple par export BibTeX ou par l'envoi de courriels.
D'autres points techniques sont discutés, notamment l'impact d'une édition
effectuée par un utilisateur sur la présentation du papier concerné dans la
liste de publications d'autres utilisateurs.

\paragraph{Gestion des articles scientifiques manquants.} Raito Bezarius sonde
l'AG quant à la réponse opportune à apporter aux gestionnaires de dépôts qui
demandent à Dissemin de les indexer. Antonin Delpeuch rappelle que ces dépôts
doivent être inscrits dans BASE.

Marie Farge soulève la question de l'identification des sources en \emph{green
open access} et en \emph{gold open access}. Elle souligne les limites de
l'approche actuelle, et s'interroge sur la pertinence de conserver ces
indicateurs, au vu des erreurs commises et de la possibilité de
mésinterprétation de ces statistiques. Antonin Delpeuch répond que cette
question est actuellement en discussion, suite à une remontée utilisateur.

Il est proposé par l'AG de permettre aux utilisateurs de Dissemin de renseigner
les détails d'articles inconnus des sources bibliographiques, pour les inclure
dans une liste de publications. Un risque serait toutefois que des utilisateurs
usurpent les publications de tiers. Marie Farge fait état de la frustration
récurrente des chercheurs, lorsqu'ils constatent que des informations
grossièrement erronées au sujet de leurs articles sont ajoutées à des
référentiels bibliographiques, le plus souvent sans contrôle et sans possibilité
pour le chercheur d'effectuer des corrections.

Antoine Amarilli soulève la question de l'indexation Web, pour extraire les
articles de recherche qui ne sont pas présents dans les sources bibliographiques
de Dissemin. Antonin Delpeuch distingue trois utilisations possibles de
l'indexation : premièrement, la validation des informations de disponibilité du
texte intégral mentionnées par nos sources bibliographiques, à effectuer par un
téléchargement manuel du fichier PDF ; deuxièmement,
un usage restreint de l'indexation, sur des dépôts
manuellement identifiés et dépourvus d'interface OAI-PMH, pour en extraire les
articles ; troisièmement, une utilisation large de l'indexation, similaire à
celle de Google Scholar, pour identifier davantage d'articles. Il souligne
néanmoins les difficultés techniques que rencontrerait une pareille indexation.

Marie Farge rappelle toutefois l'enjeu du problème des papiers manquants. Elle
indique qu'il pouvait s'avérer délicat de présenter Dissemin à des utilisateurs
potentiels, en raison du risque d'échec si le système ne répertorie aucune
publication du chercheur concerné. Elle indique que, dans certains pays, ce
problème se pose de manière récurrente.

\paragraph{Organisation centralisée ou fédérée.}
Raito Bezarius attire l'attention de l'AG sur le changement progressif de
stratégie de Dissemin, et sur ses conséquences techniques. La vision originale
de Dissemin était de proposer un système fédéré, qui pouvait être installé
indépendamment par chacun des institutions concernées. Néanmoins, étant donné la
nécessité d'accomplir des tâches de moissonnage gourmandes en ressources,
Dissemin s'oriente davantage vers un modèle centralisé, où une instance centrale
(gérée par l'association) se chargerait seule de l'exécution de ces tâches. 

Après discussions, un modèle hybride est proposé, où l'instance centrale se
chargerait des tâches lourdes ; étant bien entendu que, le code logiciel étant
disponible sous licence libre, ces tâches pourraient cependant être exécutées
indépendamment par tout utilisateur intéressé. L'instance centrale pourrait alors
proposer une interface de programmation (API) permettant à des instances
``légères'' d'accéder au résultat des opérations. Ces instances permettraient
aux institutions intéressées de pouvoir facilement installer Dissemin localement
sur leurs machines, et leur permettrait également de personnaliser facilement
l'interface, de l'intégrer à des portails d'authentification locaux, et d'en
modifier le design conformément à leur image de marque. La question de la
politique d'accès à l'API est discutée : celle-ci pourrait être payante, à titre
de participation aux frais. Antoine Amarilli mentionne la possibilité de mise à
disposition par les instances légères d'une partie de leurs ressources, en
contrepartie d'un accès à l'API, afin d'être mises à contribution pour les
tâches lourdes.

L'AG évoque également la position à tenir pour répondre aux demandes
d'universités ou de documentalistes qui désirent mettre en place une
installation de Dissemin. À l'heure actuelle, il n'est pas décidé d'encourager
cette pratique, étant donné le caractère préliminaire du prototype actuel
d'implémentation. Marie Farge rappelle toutefois que Dissemin répond à une
véritable demande de la part de nombreux acteurs.

\paragraph{OAbot.}
Antonin Delpeuch rappelle que l'objet d'OAbot est de travailler à l'édition
automatisés des liens vers des articles scientifiques sur l'encyclopédie
collaborative en ligne Wikipédia, afin de privilégier les versions disponibles
en libre accès. Antonin fait brièvement état des réunions entreprises à ce sujet
avec les wikipédiens intéressés, et avec CUNY, notamment par rapport à la
proposition d'intégrer à OAI-PMH des informations plus riches sur la licence des
articles (et non seulement sur leur disponibilité). L'enjeu de visibilité sur
Wikipédia est notamment de nature à encourager les universités à œuvrer pour
ajouter des liens vers les articles de leurs chercheurs sur Wikipédia.

Antonin rappelle l'état actuel de la discussion, qui porte notamment sur le
graphisme de l'icône indiquant les copies en libre accès dans les références de
Wikipédia. Il mentionne les difficultés techniques liées à l'édition de patrons
(\emph{templates}) utilisés sur de nombreuses pages Wikipédia, qui nécessite de
faire appel à un administrateur.

\paragraph{OA Week.}
L'AG s'interroge sur la possibilité d'organiser des événements dans le cadre de
l'OA Week. Les membres expriment des réserves quant à leurs disponibilités à
cette fin, et quant à la proximité de l'échéance ; la possibilité que Dissemin
s'intègre au programme d'autres institutions est notamment mise en avant.
L'AG discute toutefois des possibilités d'organisation d'un hackathon autour du
libre accès, par exemple en lien avec la mission Etalab ; mais cette proposition
dépend entre autres des financements obtenus par l'association entre temps.

\paragraph{Démonstration du dépôt dans HAL.}
Antonin Delpeuch fait une démonstration à l'AG du prototype actuel d'interface
pour le dépôt dans HAL. Le champ ``Abstract'' est automatiquement rempli si
possible, et la classification de l'article est inférée à partir de celle-ci si
nécessaire. Antoine Amarilli interroge Antonin sur le champ permettant de
déposer pour le compte d'un tiers, jugé inutile dans le cas d'utilisateurs
identifiés par un compte HAL et déposant pour leur propre compte. Marie Farge
rappelle les problèmes liés aux dépôts massifs effectués pour le compte d'un
tiers.

Le champ ``Affiliations'', concernant uniquement le déposant, est autocomplété à
partir de ce que l'utilisateur saisit. Marie Farge insiste sur la nécessité de
permettre aux utilisateurs d'ajouter leurs propres institutions, par exemple par
saisie de texte libre.

Pablo Rauzy soulève la question de la modération effectuée par HAL, qui implique
que le dépôt est susceptible d'échouer après un certain délai. En conséquence,
il apparaît nécessaire de proposer un retour aux utilisateurs sur le résultat de
leurs dépôts dans HAL. Antoine Amarilli pose également la question de
modifications ultérieures que le déposant, ou ses coauteurs, pourraient vouloir
effectuer directement dans HAL. Antonin Delpeuch et Raito Bezarius rappellent
que HAL indique à Dissemin le résultat de l'action par courriel, et transmet
également un mot de passe permettant de prendre contrôle de l'article sur HAL.

Deux solutions, possiblement complémentaires l'une de l'autre, sont étudiées par l'AG. En premier lieu, Dissemin
peut faire suivre aux utilisateurs par courriel le résultat de l'action et le
mot de passe communiqué par HAL. En second lieu, ces informations (et, dans
l'intervalle, le nombre d'articles en cours de modération dans HAL) peuvent être
communiquées à l'utilisateur par un bandeau sur le site. Il est mentionné que
les mots de passe envoyés par HAL constitueraient des données sensibles et
devraient être traitées en conséquence.

\paragraph{Open Science Conference.}
Marie Farge indique que Raghavendra Gadagkar donnera une conférence à l'ENS le
12 décembre, et que des manifestations liés à Dissemin pourraient s'intégrer à
cet événement. Notamment, si Dissemin obtient le financement PSL Explore, de
telles manifestations pourraient correspondre aux volets des animations liées au
libre accès, que C.A.P.S.H.\ s'engagerait alors à organiser à l'ENS.

\paragraph{Listes de diffusion.}
Antonin Delpeuch propose un récapitulatif des abonnés à la liste de diffusion
interne de l'équipe, et procède à un rappel des bonnes pratiques de
communication électronique liées à leur usage. Pablo Rauzy mentionne la
possibilité d'apposer un préfixe à l'objet des messages transmis par la liste,
afin d'éviter toute confusion.

\paragraph{Point sur la loi Lemaire.}
L'AG rappelle les enjeux de la loi Lemaire liés au libre accès et aux objectifs
de l'association, et s'informe de son progrès actuel dans le processus
législatif.

\paragraph{Dissemin et données ouvertes.}
Raito Bezarius interroge l'AG sur les données gouvernementales dont C.A.P.S.H.\
pourrait avoir besoin, en vue de demander leur ouverture. Marie Farge mentionne
les fiches bibliographiques soumises par les personnels CNRS dans le cadre de
leurs activités. L'AG discute de la
possibilité d'en obtenir communication par le biais d'un recours CADA, mais émet
des réserves liées au caractère nominatif de ces documents.

\paragraph{Plateformes de discussion.}
Raito Bezarius interroge l'AG quant à la possibilité d'utiliser de nouvelles
plateformes de messagerie instantanée en ligne pour Dissemin, qui
seraient plus faciles d'accès que la plateforme IRC actuellement utilisée. La
question de l'accès à IRC est posée par certains membres de l'AG. Plusieurs
solutions techniques sont proposées. La nécessité de diviser le canal de
discussion IRC actuel en plusieurs canaux thématiques est également discutée.

\paragraph{Organisation de réunions régulières.}
L'AG discute de l'opportunité d'organiser de réunions périodiques, sujettes aux
disponibilités de ses membres et à leurs contraintes géographiques. L'AG se
décide pour une réunion mensuelle, le premier jeudi du mois à 17h30.

\bigskip

L'ordre du jour étant épuisé, la séance est levée à 21h50.
\end{document}

