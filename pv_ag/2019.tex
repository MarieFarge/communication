\documentclass[a4paper]{article}

\usepackage[official]{eurosym}
\usepackage[french]{babel}

\title{Procès-verbal de l'assemblée générale 2019}
\date{30 septembre 2019}

\begin{document}

\maketitle

L'assemblée générale ordinaire du Comité pour l'Accessibilité aux Publications en Sciences et Humanités (C.A.P.S.H.) s'est tenue le 20 septembre 2019 à l'École normale supérieure, 24 rue Lhomond.

Sont présents:
\begin{itemize}
  \item Antoine Amarilli
  \item Antonin Delpeuch
  \item Lucas Verney
  \item Marie Farge
  \item Patricia Mirabile
\end{itemize}

La séance est présidée par Patricia et est ouverte à 18h23.

\section{Bilan moral}

Antonin résume les activités de l'association au cours de l'année passée.
L'activité a été concentrée autour de la collaboration avec le projet OpenIng, mené
par les universités techniques de Braunschweig, de Darmstadt et l'université de 
Stuttgart. Ces universités développent des connecteurs permettant d'utiliser
Dissemin pour déposer dans leurs archives ouvertes, et contribuent à la plateforme
à un rythme soutenu depuis le début de l'année civile.
Marie mentionne qu'il existe un groupe de 9 universités techniques qui est peut-être
le cadre plus général du projet.
Le projet OpenIng se terminera à la fin de l'année civile.
L'assemblée discute de la nouvelle page d'accueil de la plateforme qui est bien reçue.
L'ajout des dépôts récents est apprécié, mais Patricia et Antoine notent un risque 
que la plateforme annonce son inactivité si personne ne dépose pendant quelques jours.
Marie suggère de mettre les statistiques d'accès aux papiers avant les derniers dépôts, et
souhaiterait voir réapparaître la grue en papier (qui forme le logo de Dissemin) sur 
la page principale.
Lucas mentionne l'implémentation des lettres de dépôt, qui sont générées automatiquement
 par la plateforme. Il estime que c'est une fonctionnalité utile pour la plateforme à l'avenir car c'est une contrainte réelle de beaucoup d'universités.
Antonin estime que cette collaboration est une excellente opportunité pour la plateforme
et que nous serions heureux de la poursuivre dans le futur, avec OpenIng ou d'autres universités. Lucas regrette que le projet prenne fin dans quelques mois alors que le développement est devenu très dynamique et efficace.
Antoine mentionne le travail de Charles Paperman sur une nouvelle architecture pour l'ingestion de publications. Antonin confirme que c'est une piste prometteuse mais qu'il ne peut pas s'engager personnellement à implémenter le prototype dans la
plateforme.
Le nombre de dépôts effectués via Dissemin et publiés dans les dépôts est
de 500 659 entre le 1er septembre 2018 et le 1er septembre 2019.

Le bilan moral est adopté à l'unanimité.

\section{Bilan financier}

Lors du dernier bilan financier à l'assemblée générale précédente, le 2 septembre 2018, notre compte en banque contenait 4370.54 EUR.

Dans la période du 2 septembre 2018 au 1er septembre 2019, les mouvements ci dessous ont eu lieu:

\begin{itemize}
  \item Revenus (dons):
    \begin{itemize}
      \item Antoine Amarilli: 12 $\times$ 5 EUR = 60 EUR
      \item Pablo Rauzy: 12 $\times$ 10 EUR = 120 EUR
    \end{itemize}
   \item Dépenses
     \begin{itemize}
       \item Hébergement auprès d'Online SAS:  7 $\times$ 86,36 EUR + 203,44 EUR + 97,16 EUR + 189,66 EUR + 2 $\times$ 145,15 EUR = 1385,08 EUR
     \end{itemize}
\end{itemize}
    
Le 1er septembre 2019 notre compte en banque contenait 3165,46 EUR

Il n'y a que 0.03 EUR dans le compte Liberapay du projet et il n'y a plus de dons
via cette plateforme depuis le changement de système de paiement sous-jacent.

Dans le projet PSL Explore, il restait 9052.32 EUR au 20 juin 2019. Depuis, 911,86 EUR ont été dépensés. Pourtant, le compte indique qu'il reste 3922.64~EUR. Marie va voir Hervé Le Dréant pour régler ce problème.

Le bilan financier est adopté à l'unanimité

\section{Viabilité financière à plus long terme}

Antoine explique qu'avec les dépenses annuelles actuelles de 1811,40 EUR, la plateforme peut continuer à fonctionner jusqu'en Avril 2021.
Antonin mentionne qu'il est possible de répercuter des coûts d'hébergement sur le projet OpenIng. L'assemblée propose de facturer 2 ans d'hébergement (la durée du projet OpenIng) sur la base des coûts mensuels actuels, ce qui représente 3483.60 EUR.

\section{Demandes de financement}

Antonin mentionne les appels à projet de Paris Sciences et Lettres (PSL) et de l'Agence Nationale de la Recherche (ANR) auxquels  Dissemin a candidaté.
Pour PSL, il nous a été reproché un manque d'ambition commerciale qui assurerait une viabilité à long terme. Pour l'ANR, nous avons été jugés hors cadre, ne mettant pas assez l'accent sur l'accès aux données de recherche.
Il est envisagé de candidater à NLnet et à l'appel à projet du Conseil pour la Science Ouverte (COSO).
Antoine mentionne la possibilité de candidater au Google Summer of Code ou la proposition de stages à des étudiants, sous réserve que des développeurs soient disponibles pour encadrer.

\section{Déductibilité fiscale des dons}
Antoine explique qu'après modification des statuts lors de la dernière assemblée générale extraordinaire, l'association a obtenu l'autorisation de délivrer des reçus de dons donnant droit à des réductions d'impôt (rescrit fiscal).
Se pose la question de l'émission concrète de ces reçus ?
Marie propose de le faire un mois avant la période de déclaration d'impôts en France, ce qui voudrait dire en mars 2020 pour la première fois. Cela couvrirait les dons sur cette année civile. Antoine mentionne qu'il faudra aussi ajuster le site internet de
l'association pour expliquer aux donateurs comment recevoir ces rescrits fiscaux.
Antonin rappelle que la plateforme PayPlug utilisée pour les dons par carte bancaire
permet de contacter les donateurs par email.

\section{Renouvellement du conseil d'administration}

Antonin rappelle le rôle du conseil d'administration dans l'association.
Suite aux manifestations d'intérêt lors de discussions précédant l'assemblée générale,
il propose de remplacer Thomas Bourgeat et Pablo Rauzy par Patricia et Lucas.

Ce changement est accepté à l'unanimité.

\section{Mise à jour des membres actifs et associés}

Pablo Rauzy devient un membre actif, les autres membres sont des membres
associés.
  
\section{Discussions sur le renouvellement de l'abonnement national aux collections d'Elsevier}

L'assemblée discute de leurs avis personnels sur le renouvellement de l'accord national avec Elsevier. L'assemblée note les retours positifs sur la prise de position de l'association par le biais d'une lettre ouverte qui a été relayée dans la presse.

\section{Priorités de l'association pour l'année à venir}

L'assemblée note le manque de main d'œuvre pour maintenir et faire évoluer la plateforme,
et la difficulté de faire la promotion d'un outil dont on connait des limitations.
L'amélioration de HAL qui a moins besoin d'une surcouche pour rendre le dépôt plus facile
rend le projet moins crucial en France. La question du rôle de Dissemin dans le futur se pose. Il est noté que la plateforme continue à proposer des services que HAL n'est pas susceptible de fournir dans le futur proche, en recensant les publications à
partir de sources de métadonnées externes.
Marie note l'intérêt soutenu pour effectuer l'inventaire de publications dans une institution donnée.
Lucas suggère de reléguer le déduplication de publications à un service externe comme OpenAIRE ou Unpaywall, proposition qui est reçue favorablement par l'assemblée.
Antoine suggère de n'utiliser que HAL pour détecter la disponibilité des papiers, dans une optique où Dissemin se concentrerait sur le dépôt dans cette archive. Antonin note que cela couperait la plateforme de sa pertinence à l'étranger, notamment pour le
projet OpenIng.
Marie encourage l'utilisation de Dissemin pour déposer dans un réseau décentralisé d'archives ouvertes, et donc de conserver une assiette plus large de dépôts moissonnés.
L'assemblée note le choix entre maintenir la plateforme dans son état actuel (en finançant les dépenses de serveur), la développer activement (ce qui requiert un autre ordre de grandeur de financement) ou d'envisager une fermeture du service. Patricia note
la possibilité de co-financer du développement avec une institution partenaire.

L'ordre du jour étant épuisé, l'assemblée est levée à 21h00.

\end{document}

