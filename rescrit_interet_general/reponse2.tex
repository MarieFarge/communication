\documentclass[11pt]{lettre}

\usepackage[utf8]{inputenc}
\usepackage[francais]{babel}
\usepackage{url}
\usepackage{anysize}
\marginsize{2cm}{3cm}{3cm}{2cm}

\pagestyle{empty}

\makeatletter
\newcommand*{\NoRule}{\renewcommand*{\rule@length}{0}}
\makeatother

\begin{document}

\begin{letter}{Catherine Pasquier\\Direction départementale des finances publiques de Saône-et-Loire\\Pôle Gestion Fiscale\\Division Affaires Juridiques et Contentieuses\\Résidence Lamartine\\29 rue Lamartine\\71017 Mâcon Cedex}
\conc{Rescrit au profit d'un organisme recevant des dons : ``Comité pour l'Accessibilité aux Publications en Sciences et Humanités'' (CAPSH)}
\NoRule
\name{Antoine Amarilli}
\signature{}
\address{Antoine Amarilli (TODO adresse omise)}
\nofax
\lieu{Paris}
\telephone{}
\email{}
\opening{Madame l'inspectrice des finances publiques,}

Nous faisons suite à votre courrier daté du 25 avril (référence 2018-217, objet
  : rescrit L80C du LPF), où monsieur l'inspecteur principal des
  finances publiques Gilles Hoarau se prononçait sur l'éligibilité de notre
  association Comité pour l'Accessibilité aux Publications en Sciences et
  Humanités (CAPSH) était éligible au régime des dons prévus par les articles
  200-1.b et 238 bis.61 du Code général des impôts.

Notre compréhension de la décision de M.\ Hoarau est que notre association
  relève effectivement de ce régime, mais que cette décision était émise sous
  une unique réserve, à savoir : 

\begin{quote}
  \og Cependant, il convient de préciser que
vos statuts ne font pas état de la dévolution des biens en cas de dissolution
volontaire ou statutaire de votre association, or l'article 11 du décret du 16
août 1901 modifié par l'article 1 du décret 2017-908 du 6 mai 2017 vous fait
obligation d'indiquer dans vos statuts l'attribution des actifs de votre
association. \fg
\end{quote}

Après examen, nous reconnaissons bien volontiers que nos statuts ne faisaient
  pas état de ce point, et vous remercions d'avoir porté ce problème à notre
  connaissance. En conséquence, une assemblée générale extraordinaire a été
  convoquée au plus vite pour mettre en conformité nos statuts avec l'article 11 du décret du
  16 août 1901 modifié par l'article~1 du décret 2017-908 du 6 mai 2017, et
  ainsi lever cette réserve.

  \pagebreak

Pour attester de cela, nous vous prions de trouver ci-joint :
\nopagebreak
  \begin{itemize}
    \item Le procès-verbal de l'assemblée générale extraordinaire, qui s'est
      tenue le 24 mai
    \item La copie des nouveaux statuts adoptés par l'assemblée générale
      extraordinaire
    \item TODO le récepissé de déclaration
  \end{itemize}

Ayant ainsi levé l'unique réserve de votre courrier du 25 avril, notre
  compréhension de la situation est que notre association est donc à présent
  effectivement éligible au régime des dons prévus par les articles 200-1.b et
  238 bis.61 du Code général des impôts. Néanmoins, par souci de transparence,
  nous avons préféré vous tenir informés de la modification de nos statuts par
  le présent courrier.
  
Ainsi, si jamais cette modification de nos statuts ne suffisait pas à lever la
réserve soulevée par M.\ Hoarau, ou plus généralement si nous nous méprenons
quant à la portée de sa réponse, nous vous saurions gré de nous recontacter au
plus vite et de nous indiquer toutes réserves éventuelles qui pourraient
subsister quant à l'éligibilité de notre association au régime des dons prévus
par les articles 200-1.b et 238 bis.61 du CGI, et quant à notre capacité à
délivrer des reçus fiscaux.

En tout état de cause, les membres du comité d'administration de CAPSH en charge
de ce dossier tiennent à vous remercier ainsi que M.\ Hoarau pour la qualité et
la précision de la réponse que vous avez apportée à notre demande.

Je vous prie d'agréer, Madame l'inspectrice des finances publiques, l'assurance
de ma considération.

\vspace{2cm}

\hspace{10cm}\begin{minipage}{6cm}
%\fromsig{\includegraphics[scale=1.5]{sig.png}} \\
\fromsig{À Paris, le TODO\\Certifié exact, complet et sincère}
  \fromname{\\Membre du conseil d'administration}
\end{minipage}

\thispagestyle{empty}
\end{letter}
\end{document}

