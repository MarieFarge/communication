\documentclass[11pt]{lettre}

\usepackage[utf8]{inputenc}
\usepackage[francais]{babel}
\usepackage{url}
\usepackage{anysize}
\marginsize{2cm}{3cm}{3cm}{2cm}

\pagestyle{empty}

\makeatletter
\newcommand*{\NoRule}{\renewcommand*{\rule@length}{0}}
\makeatother

\begin{document}

\begin{letter}{Correspondant ``Associations''\\Direction départementale des finances publiques de Saône-et-Loire\\29 rue Lamartine\\71017 Mâcon Cedex}
\conc{Rescrit au profit d'un organisme recevant des dons : ``Comité pour l'Accessibilité aux Publications en Sciences et Humanités'' (CAPSH)}
\NoRule
\name{Antoine Amarilli}
\signature{}
\address{Antoine Amarilli (adresse omise)}
\nofax
\lieu{Paris}
\telephone{}
\email{}
\opening{Madame l'inspectrice des finances publiques,}

Dans votre courrier en date du 8 octobre (référence 2018-217, objet : rescrit L80C du LPF), vous nous demandez de vous transmettre des documents complémentaires pour examiner si le Comité pour l'Accessibilité au Publications en Sciences et Humanités (CAPSH) est éligible au régime des dons prévus par les articles 200-1.b et 238 bis.61 du Code général des impôts.

Nous vous prions de trouver ci-joint, en réponse à votre demande :
    
    > Les statuts datés et signés
    
TODO pintoch.ulminfo.fr/3f194388fb/statuts-signes.pdf
    
    > Les procès verbaux de délibération en assemblée générale ou extraordinaire
    
Veuillez trouver ci-joint ces documents. (TODO: PV AG 2016, 2017, 2018 et PV de l'AG constitutive avec les adresses : <55EC3ADC.3050207@delpeuch.eu>)

    > Le budget détaillant les principaux postes de recettes et de dépenses pour les 3 derniers exercices et le budget prévisionnel
    
Le budget détaillé pour les années universitaires 2015-2016, 2016-2017, 2017-2018, figure dans la section "Bilan financier" des procès-verbaux d'assemblée générale susmentionnés et joints à ce courrier.

Le budget prévisionnel pour l'année universitaire 2018-2019 est comme suit :
    
    - Recettes : dons récurrents (180 EUR/an)
    - Dépenses : 1036.32 EUR de location de serveurs (équipement informatique auprès de Online), 65.40 EUR de renouvellement de noms de domaine

Ce budget est déficitaire mais ne met pas en danger la trésorerie de l'association qui disposait (au 2 septembre 2018) d'une trésorerie de 4370.54 EUR, notamment grâce à une subvention reçue par Couperin en 2016 (voir ci-après). Nous envisagerons de solliciter d'autres telles subventions d'ici quelques années quand nous aurons fini de dépenser celle-ci.

    > Les contrats et conventions divers conclus, versements de subventions

La seule convention conclue par l'association l'a été avec le Consortium Couperin ; nous en joignons une copie (TODO). Cette convention a permis à notre association de toucher une subvention de 5400 EUR (ce montant est inférieur aux 9000 EUR prévus par la convention car le montant de la subvention a été unilatéralement changé par le Consortium après signature de la convention). L'association n'a pas conclu d'autre contrat.

pintoch.ulminfo.fr/5470a43694/convention-ccsd-signee.pdf

    > Les événements et actions organisés ou à venir

Le principal événement public organisé par l'association pour l'instant a été un "hackathon" qui a eu lieu le 8 juillet 2017 à l'École normale supérieure à Paris. Cet événement a permis de rassembler les membres de l'association et des membres intéressés du grand public pour présenter l'outil Dissemin (\url{https://dissem.in/}) développé par l'association, et pour le développement de nouvelles fonctionnalités. Il a rassemblé environ 12 personnes.

Un autre type d'action de l'association consiste en la promotion de l'outil Dissemin à divers congrès scientifiques et rassemblement autour des thématiques de l'association, à savoir l'accès ouvert aux publications scientifiques. Cette activité est exercée par certains membres du conseil d'administration dans le cadre de leurs activités professionnelles d'enseignement et de recherche, sans aide financière de l'association. Nous joignons (TODO https://association.dissem.in/index.html.fr) la liste des interventions par des membres de l'association.

Nous signalons que l'objet principal de notre association n'est pas l'organisation d'événements présentiels, mais le développement du logiciel et site Web Dissemin. S'agissant d'un outil informatique, son développement est effectué à distance par les membres de l'association, sans qu'il ne soit besoin d'organiser des événements.

    > Les rapports d'activités

Le rapport des activités de l'association figure dans les procès-verbaux d'assemblée générale susmentionnés, à la rubrique "Bilan moral".

Nous attachons également (TODO) la liste des changements récents effectués sur le code source de l'outil informatique Dissemin, telle qu'elle figure sur notre système de contrôle de versions \url{https://github.com/dissemin/dissemin/commits/master}. Ce développement informatique constitue l'activité principale de l'association.

    > Le nombre d'adhérents à ce jour ainsi que le montant de la cotisation annuelle

À ce jour, les seuls adhérents de l'association sont les membres du conseil d'administration : Antoine Amarilli, Thomas Bourgeat, Antonin Delpeuch, Marie Farge, Pablo Rauzy. D'autres personnes sont intéressés par les activités de l'association sans que nous leur ayons formellement conférés la qualité de membre (cf la liste des présents aux diverses assemblées générales). Les adhérents de l'association ne paient pas de cotisation, mais certains effectuent des dons : les 15 EUR/mois de dons reçus par l'association sont payés par Antoine Amarilli et Pablo Rauzy, de manière volontaire. En ce qui concerne les autres dons, notamment ponctuels, certains ont été versés par d'autres membres du comité d'administration, d'autres l'ont été par les utilisateurs de l'outil en ligne Dissemin, toujours de manière volontaire.

Nous tenons à attirer votre attention sur le fait que le faible nombre d'adhérents de l'association s'explique par le fait que son activité principale est le développement d'un outil informatique disponible en ligne pour tous. Ainsi, ce sont les utilisateurs de cet outil qui bénéficient des activités de l'association, et non ses seuls adhérents. Plus précisément, sur l'année universitaire 2017-2018, 1276 utilisateurs différents ont utilisé l'outil Dissemin pour déposer un article de recherche en accès libre, et nous estimons qu'au moins plusieurs milliers d'utilisateurs ont utilisé le site Dissemin sur cette période pour accéder aux publications de chercheurs. Tous ces utilisateurs bénéficient ainsi des activités de l'association, sans en être adhérents, et sans verser de cotisation ni de paiement d'aucune sorte.

\vspace{2cm}

\hspace{10cm}\begin{minipage}{6cm}
%\fromsig{\includegraphics[scale=0.5]{sig.png}} \\
\fromsig{À Paris, le TODO\\Certifié exact, complet et sincère}
  \fromname{\\Membre du conseil d'administration}
\end{minipage}

\thispagestyle{empty}
\end{letter}
\end{document}

