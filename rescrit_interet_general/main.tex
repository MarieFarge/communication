\documentclass[11pt]{lettre}

\usepackage[utf8]{inputenc}
\usepackage[francais]{babel}
\usepackage{url}
\usepackage{anysize}
\marginsize{2cm}{3cm}{3cm}{2cm}

\pagestyle{empty}

\makeatletter
\newcommand*{\NoRule}{\renewcommand*{\rule@length}{0}}
\makeatother

\begin{document}

\begin{letter}{Correspondant ``Associations''\\Direction départementale des finances publiques de Saône-et-Loire\\29 rue Lamartine\\71017 Mâcon Cedex}
\conc{Rescrit au profit d'un organisme recevant des dons}
\NoRule
\name{Antoine Amarilli}
\signature{}
\address{Antoine Amarilli TODO mon adresse}
\nofax
\lieu{Paris}
\telephone{}
\email{}
\opening{Madame, Monsieur}

  Je représente l'association loi 1901 ayant pour titre ``Comité pour
  l'Accessibilité aux Publications en Sciences et Humanités'' (CAPSH) et j'ai l'honneur
  de m'adresser à vous pour déterminer si notre association relève de l'une des
  catégories mentionnées aux articles 200 et 238 bis du code général des impôts.
  Ainsi, en vue de bénéficier de la garantie prévue à l'article L.~80~C du
  livre des procédures fiscales (LPF), et conformément au décret 2004-692 du 12
  juillet 2004, nous vous prions de trouver ci-après le modèle de demande d'avis
  relative à la mise en œuvre de cette garantie (document BOI-LETTRE-000132-20140728).

  Si des éléments complémentaires sont nécessaires afin d'apprécier notre
  situation, nous nous tenons à votre disposition pour les
  fournir, et vous invitons à nous en faire part par lettre recommandée avec
  demande d'avis de réception, conformément à l'article R*80~C-3 du LPF.

  Nous vous rappelons que, conformément à l'article R*80~C-4 du LPF, en cas de
  défaut de réponse de votre administration dans un délai de six mois à compter
  de la réception de la présente demande ou de la fourniture des compléments
  demandés, il deviendra possible pour notre association de délivrer des reçus
  fiscaux à ses donateurs ouvrant droit à crédit d'impôt.

  \newpage

  {\Large I. Identification de l'auteur de la demande}

  \textbf{Nom :} Antoine Amarilli

  \textbf{Qualité :} Membre du conseil d'administration

  TODO mes coordonnées

  \bigskip
  {\Large II. Identification de l'organisme}

  TODO Une copie des statuts est jointe à cet envoi.

  \textbf{Dénomination :} Comité pour l'Accessibilité aux Publications en Sciences
  et Humanités (CAPSH)

  \textbf{Adresse du siège social :} 34 rue de la Chanaise, 71250 Cluny

  \textbf{Objet statutaire :} la promotion de l'accès libre aux publications académiques
  et aux données de recherche, le développement et la mise en place d'une
  infrastructure logicielle pour faciliter l'accès à ces données, et le
  financement de ces activités par la collecte de dons et de fonds publics ou
  privés.

  \textbf{Affiliation :} néant

  \textbf{Imposition aux impôts commerciaux :} non

  \bigskip
  {\Large III. Composition et gestion de l'organisme }

  \textbf{Nombre de membres :} 5 personnes physiques

  \textbf{Qualité des membres :} les 5 membres sont tous membres du comité
  d'administration, et sont à ce titre convoqués aux assemblées générales, où
  ils ont le droit de vote.

  \textbf{Noms, adresses, professions des dirigeants :}
  \begin{itemize}
    \item TODO
  \end{itemize}

  \textbf{Salariés :} néant

  \bigskip
  {\Large IV. Activités exercées}

  \textbf{Lieu d'exercice des activités :} domicile des membres, établissements
  d'enseignement supérieur et de recherche, événements et congrès scientifiques

  \textbf{Activités exercées :} À titre permanent :

  \begin{itemize}
    \item Maintenance du site Web \url{https://dissem.in/}
    \item Développement du logiciel Dissemin
    \item Support des utilisateurs du site Web Dissemin et du logiciel
    \item Développement d'outils liés au libre accès pour des projets de
      logiciel libre, par exemple ceux de la Wikimedia Foundation
  \end{itemize}

  À titre occasionnel :

  \begin{itemize}
    \item Présentations du logiciel Dissemin pour un public universitaire
    \item Présentations sur le libre accès pour un public universitaire
    \item Organisation d'événements autour des activités de l'association
      (``hackathons'').
  \end{itemize}

  \textbf{Modalité d'exercice :} Le site Web Dissemin est accessible à tous, et
  l'accès au site est gratuit. Le code source du logiciel Dissemin est
  accessible à tous et distribué à titre gratuit sous une licence libre. Le
  support des utilisateurs du site Web et du logiciel Dissemin est proposé à
  tous à titre gratuit. Les présentations réalisées par les membres de
  l'association au titre de leurs fonctions s'adressent au public des événements
  concernés : les exposés sont effectués par les membres à titre gratuit. Le
  travail de développement effectué par les membres de l'association pour
  d'autres projets bénéficient ces autres projets, et sont effectués à titre
  gratuit. L'accès aux événements présentant les activités de l'association est
  ouvert au public et gratuit.

  \textbf{Description des projets en cours :}

  \begin{itemize}
    \item Site Web et logiciel Dissemin \url{https://dissem.in/}
    \item Logiciel OAbot pour l'ajout de liens vers des copies en libre accès
      des travaux scientifiques cités sur Wikipedia.
  \end{itemize}

  \bigskip
  {\Large V. Ressources de l'organisme}

  \textbf{Dons :} 253.23~EUR en 2017--2018; depuis la création de l'association,
  le montant des dons sur une année académique n'a jamais dépassé 1000~EUR.

  \textbf{Autres :}

  \begin{itemize}
    \item \textbf{Cotisations :} néant
    \item \textbf{Subventions :}
      \begin{itemize}
        \item 5400 EUR versée par le consortium unifié des
      établissements universitaires et de recherche pour l'accès aux
      publications numériques (COUPERIN) dans le cadre d'un développement
      informatique sur le logiciel Dissemin et sur la plateforme Hyper articles
          en ligne (HAL) développée par le Centre pour la communication
          scientifique directe (CCSD) du CNRS
      \item 15\,440 EUR dans le cadre d'un appel à projet de la plateforme
        PSL-Explore de l'université de recherche Paris-Sciences-et-Lettres,
          portant sur le développement de la plateforme Dissemin et la diffusion
          de contenu numériques ; TODO mentionner ?
      \end{itemize}
    \item \textbf{Ventes :} néant
    \item \textbf{Prestations :} néant
  \end{itemize}

  \textbf{Existence d'un secteur lucratif :} Non.

  \bigskip
  {\Large VI. Observations complémentaires}

  La principale activité de l'association est de maintenir un logiciel et un
  site Web gratuits, disponible sur \url{https://dissem.in}, qui s'adresse aux chercheurs du monde
  entier et leur permet de déposer facilement leurs articles scientifiques en
  libre accès pour qu'ils puissent être lus gratuitement par le public. Le code
  source de ce logiciel est gratuit et sous licence libre. Les membres
  de l'association exercent leur fonction à titre bénévole sans percevoir de
  rémunération, TODO les seules indemnités perçues étant des frais de déplacement
  pour se rendre à des conférences scientifiques où présenter l'outil, ou pour
  organiser des réunions pour le développement du logiciel, ou des assemblées
  générales.

  À l'heure actuelle, les ressources de l'utilisation ne sont utilisées que pour
  le paiement de son infrastructure informatique (location de serveurs dédiés et
  de nom de domaine) qui sont intégralement consacrés aux activités de
  l'association.

  Veuillez agréer, Madame, Monsieur, l'expression de mes salutations distinguées.

\vspace{2cm}

\hspace{10cm}\begin{minipage}{6cm}
%\fromsig{\includegraphics[scale=0.5]{sig.png}} \\
\fromsig{À Paris, le TODO DATE\\Certifié exact, complet et sincère}
  \fromname{\\Membre du conseil d'administration}
\end{minipage}

\thispagestyle{empty}
\end{letter}
\end{document}

